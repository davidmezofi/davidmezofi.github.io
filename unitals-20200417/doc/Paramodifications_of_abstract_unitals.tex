% generated by GAPDoc2LaTeX from XML source (Frank Luebeck)
\documentclass[a4paper,11pt]{report}

\usepackage[top=37mm,bottom=37mm,left=27mm,right=27mm]{geometry}
\sloppy
\pagestyle{myheadings}
\usepackage{amssymb}
\usepackage[utf8]{inputenc}
\usepackage{makeidx}
\makeindex
\usepackage{color}
\definecolor{FireBrick}{rgb}{0.5812,0.0074,0.0083}
\definecolor{RoyalBlue}{rgb}{0.0236,0.0894,0.6179}
\definecolor{RoyalGreen}{rgb}{0.0236,0.6179,0.0894}
\definecolor{RoyalRed}{rgb}{0.6179,0.0236,0.0894}
\definecolor{LightBlue}{rgb}{0.8544,0.9511,1.0000}
\definecolor{Black}{rgb}{0.0,0.0,0.0}

\definecolor{linkColor}{rgb}{0.0,0.0,0.554}
\definecolor{citeColor}{rgb}{0.0,0.0,0.554}
\definecolor{fileColor}{rgb}{0.0,0.0,0.554}
\definecolor{urlColor}{rgb}{0.0,0.0,0.554}
\definecolor{promptColor}{rgb}{0.0,0.0,0.589}
\definecolor{brkpromptColor}{rgb}{0.589,0.0,0.0}
\definecolor{gapinputColor}{rgb}{0.589,0.0,0.0}
\definecolor{gapoutputColor}{rgb}{0.0,0.0,0.0}

%%  for a long time these were red and blue by default,
%%  now black, but keep variables to overwrite
\definecolor{FuncColor}{rgb}{0.0,0.0,0.0}
%% strange name because of pdflatex bug:
\definecolor{Chapter }{rgb}{0.0,0.0,0.0}
\definecolor{DarkOlive}{rgb}{0.1047,0.2412,0.0064}


\usepackage{fancyvrb}

\usepackage{mathptmx,helvet}
\usepackage[T1]{fontenc}
\usepackage{textcomp}


\usepackage[
            pdftex=true,
            bookmarks=true,        
            a4paper=true,
            pdftitle={Written with GAPDoc},
            pdfcreator={LaTeX with hyperref package / GAPDoc},
            colorlinks=true,
            backref=page,
            breaklinks=true,
            linkcolor=linkColor,
            citecolor=citeColor,
            filecolor=fileColor,
            urlcolor=urlColor,
            pdfpagemode={UseNone}, 
           ]{hyperref}

\newcommand{\maintitlesize}{\fontsize{50}{55}\selectfont}

% write page numbers to a .pnr log file for online help
\newwrite\pagenrlog
\immediate\openout\pagenrlog =\jobname.pnr
\immediate\write\pagenrlog{PAGENRS := [}
\newcommand{\logpage}[1]{\protect\write\pagenrlog{#1, \thepage,}}
%% were never documented, give conflicts with some additional packages

\newcommand{\GAP}{\textsf{GAP}}

%% nicer description environments, allows long labels
\usepackage{enumitem}
\setdescription{style=nextline}

%% depth of toc
\setcounter{tocdepth}{1}





%% command for ColorPrompt style examples
\newcommand{\gapprompt}[1]{\color{promptColor}{\bfseries #1}}
\newcommand{\gapbrkprompt}[1]{\color{brkpromptColor}{\bfseries #1}}
\newcommand{\gapinput}[1]{\color{gapinputColor}{#1}}


\begin{document}

\logpage{[ 0, 0, 0 ]}
\begin{titlepage}
\mbox{}\vfill

\begin{center}{\maintitlesize \textbf{ Paramodifications of abstract unitals \mbox{}}}\\
\vfill

\hypersetup{pdftitle= Paramodifications of abstract unitals }
\markright{\scriptsize \mbox{}\hfill  Paramodifications of abstract unitals  \hfill\mbox{}}
{\Huge \textbf{ GAP implementation and examples worksheet \mbox{}}}\\
\vfill

{\Huge  0.1 \mbox{}}\\[1cm]
{ 8 March 2020 \mbox{}}\\[1cm]
\mbox{}\\[2cm]
{\Large \textbf{ G{\a'a}bor P. Nagy, D{\a'a}vid Mez{\H o}fi \mbox{}}}\\
\hypersetup{pdfauthor= G{\a'a}bor P. Nagy, D{\a'a}vid Mez{\H o}fi }
\end{center}\vfill

\mbox{}\\

\noindent \textbf{Address: }\begin{minipage}[t]{8cm}\noindent
 Bolyai Institute of the University of Szeged (Hungary) \\
 Department of Algebra of the Budapest University of Technology and Economics
(Hungary) \end{minipage}
\end{titlepage}

\newpage\setcounter{page}{2}
{\small 
\section*{Abstract}
\logpage{[ 0, 0, 1 ]}
 GAP implementation of paramodifications of abstract unitals. With small
modifications, it must work for other 2-designs as well. \mbox{}}\\[1cm]
{\small 
\section*{Copyright}
\logpage{[ 0, 0, 2 ]}
 2020 G{\a'a}bor P. Nagy \mbox{}}\\[1cm]
{\small 
\section*{Acknowledgements}
\logpage{[ 0, 0, 3 ]}
 Support provided from the National Research, Development and Innovation Fund
of Hungary, financed under the 2018-1.2.1-NKP funding scheme, within the SETIT
Project 2018-1.2.1-NKP-2018-00004. Partially supported by OTKA grants 119687
and 115288. \mbox{}}\\[1cm]
\newpage

\def\contentsname{Contents\logpage{[ 0, 0, 4 ]}}

\tableofcontents
\newpage

     
\chapter{\textcolor{Chapter }{Commands for paramodifications}}\label{Chapter_Commands_for_paramodifications}
\logpage{[ 1, 0, 0 ]}
\hyperdef{L}{X7B51F4D982842F51}{}
{
  
\section{\textcolor{Chapter }{Prerequisites}}\label{Chapter_Commands_for_paramodifications_Section_Prerequisites}
\logpage{[ 1, 1, 0 ]}
\hyperdef{L}{X7B4649CF7B7CFAA1}{}
{
  The \textsf{GAP} packages \textsf{UnitalSZ} and \textsf{GRAPE} are necessary. 
\begin{Verbatim}[commandchars=!@|,fontsize=\small,frame=single,label=Example]
  !gapprompt@gap>| !gapinput@LoadPackage( "UnitalSZ", false );|
  !gapprompt@gap>| !gapinput@LoadPackage( "grape", false );|
\end{Verbatim}
 You can use the info class \texttt{InfoParamod} to get extra information. 
\begin{Verbatim}[commandchars=!@|,fontsize=\small,frame=single,label=Example]
  !gapprompt@gap>| !gapinput@DeclareInfoClass( "InfoParamod" ); |
\end{Verbatim}
 

\subsection{\textcolor{Chapter }{AllRegularBlockColorings}}
\logpage{[ 1, 1, 1 ]}\nobreak
\hyperdef{L}{X852D1E7D8343F655}{}
{\noindent\textcolor{FuncColor}{$\triangleright$\enspace\texttt{AllRegularBlockColorings({\mdseries\slshape bls, k, gr})\index{AllRegularBlockColorings@\texttt{AllRegularBlockColorings}}
\label{AllRegularBlockColorings}
}\hfill{\scriptsize (function)}}\\
\textbf{\indent Returns:\ }
the list of block colorings of the set \mbox{\texttt{\mdseries\slshape bls}} of blocks, with \mbox{\texttt{\mdseries\slshape k}} colors and color classes of size \texttt{|bls|/k}. The colorings are returned as GAP transformation objects from \texttt{[1..Size(bls)]} to \texttt{[1..k]}. 



 Let $P$ be a set, $B$ a set of subsets (called \emph{blocks}), and $k$ a positive integer. The map $\chi:B\to \{1,\ldots,k\}$ is a block coloring if $\chi(b)=\chi(b')$ implies $b\cap b'=\emptyset$ for any $b,b' \in B$. The block coloring in \emph{regular}, if each color class has size $|B|/k$. The last argument \mbox{\texttt{\mdseries\slshape gr}} must be a block preserving permutation of the set $P$ of points. }

 }

 
\section{\textcolor{Chapter }{Paramodifications}}\label{Chapter_Commands_for_paramodifications_Section_Paramodifications}
\logpage{[ 1, 2, 0 ]}
\hyperdef{L}{X7E8B6954810EDCF9}{}
{
  

\subsection{\textcolor{Chapter }{ParamodificationOfUnital}}
\logpage{[ 1, 2, 1 ]}\nobreak
\hyperdef{L}{X8779000580CECE00}{}
{\noindent\textcolor{FuncColor}{$\triangleright$\enspace\texttt{ParamodificationOfUnital({\mdseries\slshape u, b, chi})\index{ParamodificationOfUnital@\texttt{ParamodificationOfUnital}}
\label{ParamodificationOfUnital}
}\hfill{\scriptsize (function)}}\\
\textbf{\indent Returns:\ }
the paramodification of the unital \mbox{\texttt{\mdseries\slshape u}} with respect to the block \mbox{\texttt{\mdseries\slshape b}} and regular block coloring \mbox{\texttt{\mdseries\slshape chi}} with \texttt{|b|} colors. 



 The coloring \mbox{\texttt{\mdseries\slshape chi}} must be given as a GAP transformation object from \texttt{[1..q\texttt{\symbol{94}}2*(q\texttt{\symbol{94}}2-q+1)]} to \texttt{[1..q+1]}, where \texttt{q} is the order of \mbox{\texttt{\mdseries\slshape u}}. This function has a slightly faster non-checking version \texttt{ParamodificationOfUnitalNC}. }

 

\subsection{\textcolor{Chapter }{ParamodificationsOfUnitalWithBlock}}
\logpage{[ 1, 2, 2 ]}\nobreak
\hyperdef{L}{X7A808C647B7BC8BA}{}
{\noindent\textcolor{FuncColor}{$\triangleright$\enspace\texttt{ParamodificationsOfUnitalWithBlock({\mdseries\slshape u, b})\index{ParamodificationsOfUnitalWithBlock@\texttt{ParamodificationsOfUnitalWithBlock}}
\label{ParamodificationsOfUnitalWithBlock}
}\hfill{\scriptsize (function)}}\\
\textbf{\indent Returns:\ }
all paramodifications of the unital \mbox{\texttt{\mdseries\slshape u}} with respect to the block \mbox{\texttt{\mdseries\slshape b}}. The results are reduced up to isomorphism of abstract unitals. 



 

 }

 

\subsection{\textcolor{Chapter }{AllParamodificationsOfUnital}}
\logpage{[ 1, 2, 3 ]}\nobreak
\hyperdef{L}{X86140B4786431C2A}{}
{\noindent\textcolor{FuncColor}{$\triangleright$\enspace\texttt{AllParamodificationsOfUnital({\mdseries\slshape u})\index{AllParamodificationsOfUnital@\texttt{AllParamodificationsOfUnital}}
\label{AllParamodificationsOfUnital}
}\hfill{\scriptsize (function)}}\\
\textbf{\indent Returns:\ }
all paramodifications of the unital \mbox{\texttt{\mdseries\slshape u}}. The results are reduced up to isomorphism of abstract unitals. 



 

 }

 }

 
\section{\textcolor{Chapter }{Implementations}}\label{Chapter_Commands_for_paramodifications_Section_Implementations}
\logpage{[ 1, 3, 0 ]}
\hyperdef{L}{X79F3D6D08181A543}{}
{
  
\subsection{\textcolor{Chapter }{Regular block colorings}}\label{Chapter_Commands_for_paramodifications_Section_Implementations_Subsection_Regular_block_colorings}
\logpage{[ 1, 3, 1 ]}
\hyperdef{L}{X8602C529877A70B6}{}
{
  
\begin{Verbatim}[commandchars=!@|,fontsize=\small,frame=single,label=Example]
  !gapprompt@gap>| !gapinput@InstallGlobalFunction( AllRegularBlockColorings, function( bls, nr_colors, gr )|
  !gapprompt@>| !gapinput@	local Gamma, complete_subgraphs, graph_of_cliques, colorings, ret, new_blocks, c, c_vec, i, j;|
  !gapprompt@gap>| !gapinput@	# construct the line graph and its cliques|
  !gapprompt@>| !gapinput@	Gamma := Graph( gr, bls, OnSets,|
  !gapprompt@>| !gapinput@        	function( x, y ) return x <> y and Intersection( x, y ) = [];|
  !gapprompt@gap>| !gapinput@	end );|
  !gapprompt@gap>| !gapinput@	complete_subgraphs := CompleteSubgraphs( Gamma, Size(bls)/nr_colors, 1);;|
  !gapprompt@gap>| !gapinput@	complete_subgraphs := Union( List( complete_subgraphs,|
  !gapprompt@>| !gapinput@				x -> Orbit( AutomorphismGroup( Gamma ), x, OnSets ) ) );|
  !gapprompt@gap>| !gapinput@	Info( InfoParamod, 3, "cliques of the line graph computed..." );|
  !gapprompt@gap>| !gapinput@	# construct the graph of cliques and its cliques|
  !gapprompt@>| !gapinput@	graph_of_cliques := Graph( Gamma.group, complete_subgraphs, OnSets,|
  !gapprompt@>| !gapinput@				function( x, y ) return x <> y and Intersection( x, y ) = [];|
  !gapprompt@gap>| !gapinput@				end );|
  !gapprompt@gap>| !gapinput@	colorings := CompleteSubgraphs( graph_of_cliques, nr_colors, 1 );|
  !gapprompt@gap>| !gapinput@	Info( InfoParamod, 3, Size( colorings ), " block colorings computed..." );|
  !gapprompt@gap>| !gapinput@	# construct colorings|
  !gapprompt@>| !gapinput@	ret := [];|
  !gapprompt@gap>| !gapinput@	for c in colorings do|
  !gapprompt@>| !gapinput@		c_vec:=0*[1..Size(bls)];|
  !gapprompt@gap>| !gapinput@		for i in [1..nr_colors] do |
  !gapprompt@>| !gapinput@			for j in VertexNames( graph_of_cliques )[c[i]] do|
  !gapprompt@>| !gapinput@				c_vec[ Position( bls, VertexNames( Gamma )[j] ) ] := i;|
  !gapprompt@gap>| !gapinput@			od;|
  !gapprompt@gap>| !gapinput@		od;|
  !gapprompt@gap>| !gapinput@		Add(ret, Transformation(c_vec));|
  !gapprompt@gap>| !gapinput@	od;|
  !gapprompt@gap>| !gapinput@	return ret;|
  !gapprompt@gap>| !gapinput@end );|
\end{Verbatim}
 }

 
\subsection{\textcolor{Chapter }{Paramodifications}}\label{Chapter_Commands_for_paramodifications_Section_Implementations_Subsection_Paramodifications}
\logpage{[ 1, 3, 2 ]}
\hyperdef{L}{X7E8B6954810EDCF9}{}
{
  
\begin{Verbatim}[commandchars=!@|,fontsize=\small,frame=single,label=Example]
  !gapprompt@gap>| !gapinput@InstallGlobalFunction( ParamodificationOfUnitalNC, function( u, b, chi )|
  !gapprompt@>| !gapinput@	local Cb, n_Cb, C_star_b, intact_blks, B_star;|
  !gapprompt@gap>| !gapinput@	Cb := Filtered( BlocksOfUnital( u ),|
  !gapprompt@>| !gapinput@		x -> Size( Intersection( x, b ) ) = 1 );|
  !gapprompt@gap>| !gapinput@	n_Cb := Length( Cb );|
  !gapprompt@gap>| !gapinput@	C_star_b := List( [1..n_Cb],|
  !gapprompt@>| !gapinput@		i -> Union( Difference( Cb[i], b ), [ b[i^chi] ] ) );|
  !gapprompt@gap>| !gapinput@	intact_blks := Difference( BlocksOfUnital( u ), Cb );|
  !gapprompt@gap>| !gapinput@	B_star := Union( intact_blks, C_star_b );|
  !gapprompt@gap>| !gapinput@	return AbstractUnitalByDesignBlocks( B_star );|
  !gapprompt@gap>| !gapinput@end );|
\end{Verbatim}
 
\begin{Verbatim}[commandchars=!@|,fontsize=\small,frame=single,label=Example]
  !gapprompt@gap>| !gapinput@InstallGlobalFunction( ParamodificationOfUnital, function( u, b, chi )|
  !gapprompt@>| !gapinput@	local Cb;|
  !gapprompt@gap>| !gapinput@	if not b in BlocksOfUnital( u ) then |
  !gapprompt@>| !gapinput@		Error( "argument 2 must be a block of argument 1");|
  !gapprompt@gap>| !gapinput@	fi;|
  !gapprompt@gap>| !gapinput@	Cb := Filtered( BlocksOfUnital( u ),|
  !gapprompt@>| !gapinput@                   x -> Size( Intersection( x, b ) ) = 1 );|
  !gapprompt@gap>| !gapinput@	Cb := List( Cb, x -> Difference( x, b) );|
  !gapprompt@gap>| !gapinput@	if not ForAll( Combinations( [1..Size(Cb)], 2 ), p ->|
  !gapprompt@>| !gapinput@			Intersection(Cb{p})=[] or |
  !gapprompt@>| !gapinput@			(p[1]^chi<>p[2]^chi)|
  !gapprompt@>| !gapinput@		) then Error( "argument 3 is not a proper block coloring" ); |
  !gapprompt@gap>| !gapinput@	fi;|
  !gapprompt@gap>| !gapinput@	return ParamodificationOfUnitalNC( u, b, chi );|
  !gapprompt@gap>| !gapinput@end );|
\end{Verbatim}
 }

 
\subsection{\textcolor{Chapter }{Paramodifications with respect to a block}}\label{Chapter_Commands_for_paramodifications_Section_Implementations_Subsection_Paramodifications_with_respect_to_a_block}
\logpage{[ 1, 3, 3 ]}
\hyperdef{L}{X86208050795C6640}{}
{
  
\begin{Verbatim}[commandchars=!@|,fontsize=\small,frame=single,label=Example]
  !gapprompt@gap>| !gapinput@InstallGlobalFunction( ParamodificationsOfUnitalWithBlock, |
  !gapprompt@>| !gapinput@function( u, b )|
  !gapprompt@>| !gapinput@	local q, Cb, b_stab, new_unitals, all, allchibmod, i, isom_class, colorings;|
  !gapprompt@gap>| !gapinput@	if not b in BlocksOfUnital( u ) then |
  !gapprompt@>| !gapinput@		Error( "argument 2 must be a block of argument 1");|
  !gapprompt@gap>| !gapinput@	fi;|
  !gapprompt@gap>| !gapinput@	q := Order( u );|
  !gapprompt@gap>| !gapinput@	Cb := Filtered( BlocksOfUnital( u ),|
  !gapprompt@>| !gapinput@		x -> Size( Intersection( x, b ) ) = 1 );|
  !gapprompt@gap>| !gapinput@	# Important: keep the order from BlocksOfUnital|
  !gapprompt@>| !gapinput@	Cb := List( Cb, x -> Difference( x, b ) ); |
  !gapprompt@gap>| !gapinput@	# compute all colorings|
  !gapprompt@>| !gapinput@	b_stab := Stabilizer( AutomorphismGroup( u ), b, OnSets );|
  !gapprompt@gap>| !gapinput@	colorings := AllRegularBlockColorings( Cb, q + 1, b_stab );|
  !gapprompt@gap>| !gapinput@	Info( InfoParamod, 1, Size( colorings ), " coloring(s) for the given unital-block pair computed..." );|
  !gapprompt@gap>| !gapinput@	new_unitals := List( colorings, c -> ParamodificationOfUnitalNC( u, b, c ) );|
  !gapprompt@gap>| !gapinput@	# reduction up to isomorphism|
  !gapprompt@>| !gapinput@	all := [1..Length( new_unitals )];|
  !gapprompt@gap>| !gapinput@	allchibmod := [];|
  !gapprompt@gap>| !gapinput@	while all <> [] do|
  !gapprompt@>| !gapinput@		i := Remove( all );|
  !gapprompt@gap>| !gapinput@		isom_class := Filtered( all, x -> Isomorphism( new_unitals[i],|
  !gapprompt@>| !gapinput@			new_unitals[x] ) <> fail ) ;|
  !gapprompt@gap>| !gapinput@		all := Difference( all, isom_class );|
  !gapprompt@gap>| !gapinput@		Add( allchibmod, new_unitals[i] );|
  !gapprompt@gap>| !gapinput@	od;|
  !gapprompt@gap>| !gapinput@	return allchibmod;|
  !gapprompt@gap>| !gapinput@end );|
\end{Verbatim}
 }

 
\subsection{\textcolor{Chapter }{All paramodifications of a unital}}\label{Chapter_Commands_for_paramodifications_Section_Implementations_Subsection_All_paramodifications_of_a_unital}
\logpage{[ 1, 3, 4 ]}
\hyperdef{L}{X7C5ED22881441375}{}
{
  
\begin{Verbatim}[commandchars=!@|,fontsize=\small,frame=single,label=Example]
  !gapprompt@gap>| !gapinput@InstallGlobalFunction( AllParamodificationsOfUnital, function( u )|
  !gapprompt@>| !gapinput@	local blocks, rep_blocks, allchibmods, uus, b;|
  !gapprompt@gap>| !gapinput@	blocks := BlocksOfUnital( u );|
  !gapprompt@gap>| !gapinput@	rep_blocks := List( Orbits( AutomorphismGroup( u ), blocks, OnSets ),|
  !gapprompt@>| !gapinput@		      orb -> Representative( orb ) );|
  !gapprompt@gap>| !gapinput@	Info( InfoParamod, 2, Size( rep_blocks ), " block representatives for the unital computed..." );|
  !gapprompt@gap>| !gapinput@	allchibmods := [];|
  !gapprompt@gap>| !gapinput@	for b in rep_blocks do|
  !gapprompt@>| !gapinput@		uus := ParamodificationsOfUnitalWithBlock( u, b );|
  !gapprompt@gap>| !gapinput@		uus := Filtered( uus, x -> Isomorphism( x, u ) = fail and|
  !gapprompt@>| !gapinput@				ForAll( allchibmods, y -> Isomorphism( y, x ) = fail ) );|
  !gapprompt@gap>| !gapinput@		Append( allchibmods, uus );|
  !gapprompt@gap>| !gapinput@	od;|
  !gapprompt@gap>| !gapinput@	return allchibmods;|
  !gapprompt@gap>| !gapinput@end );|
\end{Verbatim}
 }

 }

 }

   
\chapter{\textcolor{Chapter }{Examples}}\label{Chapter_Examples}
\logpage{[ 2, 0, 0 ]}
\hyperdef{L}{X7A489A5D79DA9E5C}{}
{
  
\section{\textcolor{Chapter }{Computing paramodifications}}\label{Chapter_Examples_Section_Computing_paramodifications}
\logpage{[ 2, 1, 0 ]}
\hyperdef{L}{X85A7B8E583C5FA8B}{}
{
  
\begin{Verbatim}[commandchars=!@|,fontsize=\small,frame=single,label=Example]
  !gapprompt@gap>| !gapinput@LoadPackage( "UnitalSZ", false );|
  !gapprompt@gap>| !gapinput@LoadPackage( "grape", false );|
  !gapprompt@gap>| !gapinput@u:=KNPAbstractUnital(1577);|
  KNPAbstractUnital(1577)
  !gapprompt@gap>| !gapinput@AutomorphismGroup(u);|
  <permutation group with 6 generators>
  !gapprompt@gap>| !gapinput@b:=BlocksOfUnital(u)[1];|
  [ 1, 2, 5, 6, 14 ]
  !gapprompt@gap>| !gapinput@bls0:=Filtered(BlocksOfUnital(u),x->Size(Intersection(x,b))=1);;|
  !gapprompt@gap>| !gapinput@bls:=List(bls0,x->Difference(x,b));;|
  !gapprompt@gap>| !gapinput@cols:=AllRegularBlockColorings(bls,5,Group(())); time;|
  [ Transformation( [ 1, 1, 1, 4, 2, 5, 3, 5, 3, 4, 2, 2, 5, 3, 4, 4, 3, 2, 5, 5, 3, 4, 4, 5, 3, 2, 2, 1, 1, 1, 5, 3, 2, 4, 5, 2, 5,
       4, 3, 2, 2, 3, 2, 5, 3, 4, 1, 1, 1, 4, 2, 4, 4, 5, 2, 3, 2, 5, 3, 1, 1, 1, 3, 3, 2, 5, 3, 5, 4, 4, 4, 5, 1, 1, 1 ] ), 
    Transformation( [ 1, 1, 1, 1, 1, 1, 1, 1, 1, 1, 1, 1, 1, 1, 1, 4, 3, 3, 4, 2, 4, 2, 3, 3, 2, 4, 2, 5, 5, 5, 5, 3, 2, 2, 4, 5, 2,
       5, 3, 5, 3, 2, 2, 3, 5, 3, 4, 4, 4, 4, 3, 5, 3, 5, 4, 4, 5, 3, 5, 2, 2, 2, 5, 2, 4, 2, 4, 5, 5, 4, 2, 4, 3, 3, 3 ] ), 
    Transformation( [ 1, 1, 1, 1, 1, 1, 1, 1, 1, 1, 1, 1, 1, 1, 1, 4, 4, 4, 4, 5, 5, 5, 3, 3, 3, 5, 3, 3, 5, 4, 5, 3, 2, 2, 5, 3, 2,
       5, 2, 4, 2, 5, 5, 4, 4, 4, 4, 2, 5, 5, 3, 3, 2, 3, 2, 2, 5, 2, 5, 5, 3, 2, 3, 2, 4, 3, 4, 4, 4, 2, 3, 2, 4, 2, 3 ] ), 
    Transformation( [ 1, 1, 1, 1, 1, 1, 1, 1, 1, 1, 1, 1, 1, 1, 1, 4, 4, 4, 4, 4, 4, 4, 4, 4, 4, 4, 4, 4, 4, 4, 5, 3, 2, 5, 3, 2, 5,
       5, 5, 5, 5, 5, 5, 5, 5, 5, 5, 5, 5, 3, 3, 3, 3, 3, 3, 3, 3, 3, 3, 3, 3, 3, 2, 2, 2, 2, 2, 2, 2, 2, 2, 2, 2, 2, 2 ] ), 
    Transformation( [ 1, 2, 5, 5, 5, 5, 5, 2, 1, 2, 2, 1, 1, 2, 1, 4, 4, 4, 4, 2, 2, 2, 1, 1, 1, 2, 1, 1, 2, 4, 5, 3, 2, 1, 3, 5, 1,
       5, 1, 4, 1, 5, 5, 4, 4, 4, 4, 1, 5, 3, 3, 3, 3, 3, 3, 3, 3, 3, 3, 3, 3, 3, 5, 2, 4, 5, 4, 4, 4, 2, 5, 2, 4, 2, 5 ] ), 
    Transformation( [ 1, 3, 4, 4, 4, 4, 4, 3, 1, 3, 3, 1, 1, 3, 1, 4, 4, 4, 4, 5, 5, 5, 1, 1, 1, 5, 1, 1, 5, 4, 5, 3, 2, 1, 5, 2, 1,
       5, 1, 3, 1, 5, 5, 3, 3, 3, 3, 1, 5, 5, 3, 3, 4, 3, 4, 4, 5, 4, 5, 5, 3, 4, 2, 2, 2, 2, 2, 2, 2, 2, 2, 2, 2, 2, 2 ] ), 
    Transformation( [ 1, 2, 4, 4, 4, 4, 4, 2, 1, 2, 2, 1, 1, 2, 1, 4, 4, 4, 4, 2, 2, 2, 3, 3, 3, 2, 3, 3, 2, 4, 5, 3, 2, 5, 1, 3, 5,
       5, 5, 5, 5, 5, 5, 5, 5, 5, 5, 5, 5, 1, 3, 3, 4, 3, 4, 4, 1, 4, 1, 1, 3, 4, 3, 2, 1, 3, 1, 1, 1, 2, 3, 2, 1, 2, 3 ] ), 
    Transformation( [ 1, 3, 5, 5, 5, 5, 5, 3, 1, 3, 3, 1, 1, 3, 1, 4, 4, 4, 4, 4, 4, 4, 4, 4, 4, 4, 4, 4, 4, 4, 5, 3, 2, 2, 1, 5, 2,
       5, 2, 3, 2, 5, 5, 3, 3, 3, 3, 2, 5, 1, 3, 3, 2, 3, 2, 2, 1, 2, 1, 1, 3, 2, 5, 2, 1, 5, 1, 1, 1, 2, 5, 2, 1, 2, 5 ] ), 
    Transformation( [ 1, 5, 2, 1, 5, 1, 5, 2, 2, 2, 1, 2, 5, 1, 5, 4, 1, 1, 4, 1, 4, 1, 3, 3, 3, 4, 3, 3, 5, 2, 5, 3, 2, 2, 4, 3, 2,
       5, 5, 2, 5, 1, 1, 1, 2, 1, 4, 4, 4, 4, 3, 3, 5, 3, 4, 4, 5, 5, 5, 1, 3, 2, 3, 2, 4, 3, 4, 2, 2, 4, 3, 4, 1, 5, 3 ] ), 
    Transformation( [ 1, 4, 3, 1, 4, 1, 4, 3, 3, 3, 1, 3, 4, 1, 4, 4, 1, 1, 4, 1, 3, 1, 3, 3, 4, 3, 4, 5, 5, 5, 5, 3, 2, 2, 3, 5, 2,
       5, 2, 5, 2, 1, 1, 1, 5, 1, 4, 2, 3, 3, 3, 5, 2, 5, 2, 2, 5, 2, 5, 1, 4, 2, 5, 2, 4, 4, 4, 5, 5, 2, 4, 2, 1, 2, 3 ] ), 
    Transformation( [ 1, 4, 2, 1, 4, 1, 4, 2, 2, 2, 1, 2, 4, 1, 4, 4, 3, 3, 4, 5, 5, 5, 3, 3, 4, 5, 4, 1, 5, 2, 5, 3, 2, 2, 5, 1, 2,
       5, 3, 2, 3, 5, 5, 3, 2, 3, 4, 1, 5, 5, 3, 1, 3, 1, 1, 1, 5, 3, 5, 5, 4, 2, 1, 2, 4, 4, 4, 2, 2, 1, 4, 1, 3, 3, 3 ] ), 
    Transformation( [ 1, 5, 3, 1, 5, 1, 5, 3, 3, 3, 1, 3, 5, 1, 5, 4, 4, 4, 4, 2, 3, 2, 3, 3, 2, 3, 2, 1, 5, 4, 5, 3, 2, 2, 3, 1, 2,
       5, 5, 4, 5, 2, 2, 4, 4, 4, 4, 1, 3, 3, 3, 1, 5, 1, 1, 1, 5, 5, 5, 2, 2, 2, 1, 2, 4, 2, 4, 4, 4, 1, 2, 1, 4, 5, 3 ] ) ]
  342
  !gapprompt@gap>| !gapinput@List(cols,c->ParamodificationOfUnitalNC(u,b,c)); time;|
  [ AU_UnitalDesign<4>, AU_UnitalDesign<4>, AU_UnitalDesign<4>, AU_UnitalDesign<4>, AU_UnitalDesign<4>, AU_UnitalDesign<4>, 
    AU_UnitalDesign<4>, AU_UnitalDesign<4>, AU_UnitalDesign<4>, AU_UnitalDesign<4>, AU_UnitalDesign<4>, AU_UnitalDesign<4> ]
  32
  !gapprompt@gap>| !gapinput@List(cols,c->ParamodificationOfUnital(u,b,c)); time;|
  [ AU_UnitalDesign<4>, AU_UnitalDesign<4>, AU_UnitalDesign<4>, AU_UnitalDesign<4>, AU_UnitalDesign<4>, AU_UnitalDesign<4>, 
    AU_UnitalDesign<4>, AU_UnitalDesign<4>, AU_UnitalDesign<4>, AU_UnitalDesign<4>, AU_UnitalDesign<4>, AU_UnitalDesign<4> ]
  88
  !gapprompt@gap>| !gapinput@ParamodificationsOfUnitalWithBlock(u,b); time;|
  [ AU_UnitalDesign<4>, AU_UnitalDesign<4>, AU_UnitalDesign<4>, AU_UnitalDesign<4>, AU_UnitalDesign<4>, AU_UnitalDesign<4> ]
  219
  !gapprompt@gap>| !gapinput@AllParamodificationsOfUnital(u); time;|
  [ AU_UnitalDesign<4>, AU_UnitalDesign<4>, AU_UnitalDesign<4>, AU_UnitalDesign<4>, AU_UnitalDesign<4>, AU_UnitalDesign<4>, 
    AU_UnitalDesign<4>, AU_UnitalDesign<4>, AU_UnitalDesign<4>, AU_UnitalDesign<4> ]
  961
\end{Verbatim}
 }

 
\section{\textcolor{Chapter }{Para-rigidity of cyclic unitals}}\label{Chapter_Examples_Section_Para-rigidity_of_cyclic_unitals}
\logpage{[ 2, 2, 0 ]}
\hyperdef{L}{X7A2AFD31868E1D10}{}
{
  We say that a unital is \emph{para-rigid}, if all block colorings of $C(b)$ are equivalent with the trivial one. The following example shows that the
cyclic unitals of order 4 and 6 by Bagchi and Bagchi are para-rigid. 
\begin{Verbatim}[commandchars=!@|,fontsize=\small,frame=single,label=Example]
  !gapprompt@gap>| !gapinput@SetInfoLevel(InfoParamod,2);|
  !gapprompt@gap>| !gapinput@u:=BagchiBagchiCyclicUnital(4);|
  BagchiBagchiCyclicUnital(4)
  !gapprompt@gap>| !gapinput@AllParamodificationsOfUnital(u);|
  #I  2 block representatives for the unital computed...
  #I  1 coloring(s) for the given unital-block pair computed...
  #I  1 coloring(s) for the given unital-block pair computed...
  [  ]
  !gapprompt@gap>| !gapinput@u:=BagchiBagchiCyclicUnital(6);|
  BagchiBagchiCyclicUnital(6)
  !gapprompt@gap>| !gapinput@AllParamodificationsOfUnital(u);|
  #I  2 block representatives for the unital computed...
  #I  1 coloring(s) for the given unital-block pair computed...
  #I  1 coloring(s) for the given unital-block pair computed...
  [  ]
\end{Verbatim}
 }

 }

 \immediate\write\pagenrlog{["Ind", 0, 0], \arabic{page},}
\newpage
\immediate\write\pagenrlog{["End"], \arabic{page}];}
\immediate\closeout\pagenrlog
\end{document}
